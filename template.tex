% 1段組にする。
\documentclass[a4j]{jsarticle}
% 2段組にする。
%\documentclass[a4j,twocolumn]{jsarticle}

% 以下のように実行してmyjplain.bstを生成した。
% nkf jplain.bst > jplain-utf8.bst
% urlbst --doi --inlinelinks --hyperref jplain-utf8.bst myjplain.bst
\bibliographystyle{myjplain}

% 著者の所属を使う。
\usepackage{authblk}

% ハイパーリンクを使う。
\usepackage[dvipdfmx, % dvipdfmx向けのdviファイルを生成する。
pdfstartview=FitH, % dvipdfmxで生成したPDFファイルをを開くとウインドウの幅に合わせる
bookmarks=false, % しおりを作らない。
colorlinks=true,linkcolor=blue,anchorcolor=blue,citecolor=blue,filecolor=blue,menucolor=blue,runcolor=blue,urlcolor=blue % リンクの色を青くする。
]{hyperref}

% URLを扱う。
%\usepackage{url}

\title{「」について}

\author[1]{Ryo ONODERA}
\affil[1]{tetera.org}

\date{20XX-XX-XX}

\begin{document}

% \maketitleの前で定義することで、本文が2段組みの場合にもアブストラクトは1段組みにできる。
%\begin{abstract}
%\end{abstract}

% タイトルと著者と所属を印字する。
\maketitle


\section{タイトルと著者}
% タイトルは日本語訳する。
XXX 英文タイトル
~\cite{example}

XXX 日本語訳タイトル

XXX 英文著者

XXX 著者の所属とかに特筆すべきことがあれば日本語で記載する。

\section{アブストラクトの翻訳または要約}
% 全文を翻訳するにしても要約するにしても日本語にすること。
XXX


\section{気になった箇所}
% 原則は抜き書きする (コピー・アンド・ペーストするのが理想)。
% 必ずページ番号を入れること。
\begin{itemize}
 	\item	XXX (p. 101)
 	\item	XXX (pp. 111-112)
 	\item	XXX (p. 999 10行目)
\end{itemize}


\section{各章を1文で要約する}
\begin{itemize}
 	\item	これまでの研究から、著者はxxxに問題があると考えた。
		よってこの論文でxxxを調査した。
 	\item 	調べ方はxxxで、結果はxxxだった。
		その理由を著者はxxxと考えた。
 	\item	結果について、私はxxxと考えた。
\end{itemize}


\section{参考文献をたどって他の文献を読んだ記録}
\begin{itemize}
 	\item	XXX (\date{2018-10-20})
		~\cite{example-webpage}
 	\item	XXX (\date{2018-10-21})
		~\cite{example-webpage}
 	\item	XXX (\date{2018-10-22})
		~\cite{example-webpage}
\end{itemize}


\section{本文PDFファイル名}
XXX.pdf
% 著者が無料で公開している場合に記載する。
(
\url{http://www.example.com/~user/example.pdf}
)

% 参考文献を印字する。
\bibliography{myreferences}

\end{document}
