% 1段組にする。
\documentclass[a4j]{jsarticle}
% 2段組にする。
%\documentclass[a4j,twocolumn]{jsarticle}

% 以下のように実行してmyjplain.bstを生成した。
% nkf jplain.bst > jplain-utf8.bst
% urlbst --doi --inlinelinks --hyperref jplain-utf8.bst myjplain.bst
\bibliographystyle{myjplain}

% 著者の所属を使う。
\usepackage{authblk}

% ハイパーリンクを使う。
\usepackage[dvipdfmx, % dvipdfmx向けのdviファイルを生成する。
pdfstartview=FitH, % dvipdfmxで生成したPDFファイルをを開くとウインドウの幅に合わせる
bookmarks=false, % しおりを作らない。
colorlinks=true,linkcolor=blue,anchorcolor=blue,citecolor=blue,filecolor=blue,menucolor=blue,runcolor=blue,urlcolor=blue % リンクの色を青くする。
]{hyperref}

\title{「廃棄物ビジネス論」について}

\author[1]{Ryo ONODERA}
\affil[1]{tetera.org}

\date{2018-10-19}

\begin{document}

% \maketitleの前で定義することで、本文が2段組みの場合にもアブストラクトは1段組みにできる。
%\begin{abstract}
%\end{abstract}

% タイトルと著者と所属を印字する。
\maketitle


\section{タイトルと著者}
% タイトルは日本語訳する。
廃棄物ビジネス論 --- ウェイスト・マネジメント社のビジネスモデルを通して
~\cite{haikibutsu-business-ron-nagasawa}

長沢 伸也・森口 健生 著

\section{アブストラクトの翻訳または要約}
書籍のためアブストラクトなし。

\section{気になった箇所}
% 原則は抜き書きする (コピー・アンド・ペーストするのが理想)。
% 必ずページ番号を入れること。
\begin{itemize}
\item
  廃棄物処理事業は学会でもマスメディアにおいても、
  ほとんど取りあげられない。(p. 26)
\item
  注目されていないが廃棄物処理事業は多くの問題が廃棄物処理事業者自身から
  指摘されている。(p. 26)
\item
  日本では廃棄物問題は技術的課題であると考えられている(ハードウェア指向)。(p. 21)
\item
  日本の廃棄物処理企業は総じて非常に弱小である。(p. 106)
\item
  日本の廃棄物業界は、産業としては長い歴史があるにもかかわらず、新規参入が続いている
  黎明期の状態のまま現在に至っている。(p. 107)
\item
  WM (ウェイスト・マネジメント)社のように、収集・運搬・中間処理、そして最終処分を
  すべて扱う業者は少なく、さらにこれらの廃棄物処理事業の範囲を超えた
  廃棄物管理サービス業を行う処理業者は皆無である。(p. 110)
\item
  日本の廃棄物処理企業の規模は米国のそれよりはるかに小さい。(p. 112)
\item
  日本の廃棄物処理企業には高利益率を誇る企業が存在し、また全体でも比較的
  高い収益性がある。(p.112)
\item
  WM社では、請求書の二重発行や、ごみの収集し忘れ、クレームにきちんと
  対応しないといった劣悪なサービスがごみ収集会社を切り替える同期になっているとの
  調査結果が出た。(p. 154)
\item
  WM社では情報システムを導入・再構築し、顧客満足度を高めた。(p. 154)
\item
  顧客サービスに統一のマニュアルを用意し、顧客第一の思想が社内に浸透した。(pp. 154-155)
\item
  WM社の海外戦略は、「ローカルのことはローカルに任せる」という方針が
  貫かれている。(p. 158)
\item
  廃棄物事業は、総合エンジニアリング事業であるとの認識のもと、研究開発部を設けたのも
  WM社が最初であった。(p. 163)
\item
  コスト競争力を保つためにもっとも重要なことは、すべての市場において
  均一で満足できるサービスを提供することである。(pp. 169-170)
\item
  WM社は今後21年も稼働可能な埋立処分場を所有しており、他社に比べて
  決定的な競争優位を得ているため、決して薄利多売をしているわけではない。
  このため高い収益性を誇っているのである。(p. 172)
\item
  WM社は、自らの事業を総合廃棄物管理サービス業と定め、そこから顧客指向・ブランド力
  の向上と事業戦略の2つの指向性に大別できる。(p. 181)
\item
  WM社にとって、許可の申請手続きは非常に煩雑であるのに、資格要件を満たし、
  講習を受ければ誰でも収集運搬業に参入できるような許可要件の基準の低さなど、
  日本の許可制度は複雑怪奇にみえたであろう。(pp. 196-197)
\item
  大多数の廃棄物処理業者には大企業のような守るべきブランドがなく、
  費用節約のために不法投棄を犯してしまうというインセンティブ問題が発生する。(p. 199)
\item
  収集・運搬業務にも生産性を向上させる、ある程度の機械化、IT化の余地はあると考えられる。
  ただ、日本にはそのような技術開発力を持ち、事業家できる企業が残念ながら
  存在しないのである。(p. 200)
\item
  公共性の高いサービス業が守るべきブランドもない。(p. 201)
\item
  市場が統合されていないということは、成熟した市場より比較的容易に成長できる。
  したがって、技術力と資本力を有し、規模と範囲の経済性を発揮して、
  社会的信用を重視する廃棄物処理企業が出現すれば、急激に成長する可能性がある。(p. 201)
\end{itemize}


\section{各章を1文で要約する}
\begin{itemize}
\item
  廃棄物処理事業に関して、これまで論文が発表されたことがなく、マスメディアに
  注目もされていないことから、著者は廃棄物処理事業の経営について研究が必要と考え、
  アメリカ合衆国のWaste Management社のビジネスモデルを調査して、
  日本の廃棄物処理事業の課題を検討した。
\item
  日本での廃棄物処理事業者の参入障壁の低さや、零細企業が多いという著者の主張については、
  それを否定するような見解も存在する。
  ~\cite{sangyo-haikibutsu-shorigyo-no-keiei-jitsumu-daiichihoki}
\item
  結論として、日本の廃棄物処理市場の成熟のためには、大企業の出現が必要であるとした。
\end{itemize}


% 参考文献を印字する。
\bibliography{myreferences}

\end{document}
