% 1段組にする。
\documentclass[a4j]{jsarticle}
% 2段組にする。
%\documentclass[a4j,twocolumn]{jsarticle}

% 以下のように実行してmyjplain.bstを生成した。
% nkf jplain.bst > jplain-utf8.bst
% urlbst --doi --inlinelinks --hyperref jplain-utf8.bst myjplain.bst
\bibliographystyle{myjplain}

% 著者の所属を使う。
\usepackage{authblk}

% ハイパーリンクを使う。
\usepackage[dvipdfmx, % dvipdfmx向けのdviファイルを生成する。
pdfstartview=FitH, % dvipdfmxで生成したPDFファイルをを開くとウインドウの幅に合わせる
bookmarks=false, % しおりを作らない。
colorlinks=true,linkcolor=blue,anchorcolor=blue,citecolor=blue,filecolor=blue,menucolor=blue,runcolor=blue,urlcolor=blue % リンクの色を青くする。
]{hyperref}

\title{「最近の世界の石炭情勢」について}

\author[1]{Ryo ONODERA}
\affil[1]{tetera.org}

\date{2018-11-08}

\begin{document}

% \maketitleの前で定義することで、本文が2段組みの場合にもアブストラクトは1段組みにできる。
%\begin{abstract}
%\end{abstract}

% タイトルと著者と所属を印字する。
\maketitle


\section{タイトルと著者}
% タイトルは日本語訳する。
最新の世界の石炭情勢
~\cite{冨田新二2018}

冨田 新二, 岡部 修平


\section{アブストラクトの翻訳または要約}
% 全文を翻訳するにしても要約するにしても日本語にすること。
現在の世界の一次エネルギー供給の40\%は石炭によっている。
石炭は持続可能性においても経済性においても、入手可能性、入手容易性、
アクセス可能性の観点で優れている。

石炭は炭素を主成分としており、発熱量当たり多くの二酸化炭素を放出する
燃料である。
よって、環境負荷を低減する利用方法を開発することが重要であり、
この技術は「Clean Coal Technology」と呼ばれる。

世界の石炭の生産と消費の傾向は石炭価格や各国政府の政策なその急激な変動
によって変化している。


\section{気になった箇所}
% 原則は抜き書きする (コピー・アンド・ペーストするのが理想)。
% 必ずページ番号を入れること。
\begin{itemize}
\item
  世界では高品位の石炭が優先的に使われており、亜瀝青炭や褐炭といった
  低品位の石炭は利用があまり進んでいない。 (p. 137右)
\item
  低品位炭の生産割合は低下傾向である(p. 137右)
\item
  トランプ政権は石炭保護の方針と言われているが、閉鎖された石炭火力が
  再稼働するのは現実的ではなく、今後も石炭消費は横ばいからやや減少傾向で
  推移すると思われる。(p. 138左)
\item
  米国は国内消費が低迷しており輸出を志向しているものの、Wyoming州を中心
  とするPowder River丹田の亜瀝青炭は輸出するための鉄道・港の整備が不十分
  であり、現時点では輸出量を増加することが困難な状況である。(p. 138右)
\item
  特に原料炭についてはスポットでの取引量が少ないこもあり、短期間で価格が
  急上昇することとなった。(p. 139左)
\item
  売買契約もかつての長期契約からスポット契約が増えており、今後石炭ユーザーは
  今まで以上に価格変動の影響を受けやすくなると思われる。(p. 139右)
\item
  BHP Billitonは以前として石炭生産、特に豪州における原料炭生産には力を
  入れており、2017-18年度は増産を見込んでいる一方、一般炭については
  採算性の悪い炭鉱については整理を進めつつ、一定量の生産を継続していく
  模様である。
  Rio TIntoはモザンビークの石炭事業を売却、さらに今年に入り、同社子会社である
  Call and Allied (C\&A)を中国系のYancoal Australiaへ売却し、一般炭事業から
  撤退した。残るQld州の原料炭事業についても売却を検討しているという報道が
  見られる。
  Glencoreは効率の悪い炭鉱の中断・停止などを行いつつ、YancoalがC\&Aから買収した
  Hunter Valley Operationsについて、Yancoal買収後に同社から49\%権益を
  取得するなど、一般炭権益について引き続き積極的に事業展開していく方針である。
  Anglo Americanは経営悪化により豪州の一般炭炭鉱を全て売却、原料炭炭鉱も
  売却する方向で動いていたが、2016年の市況回復により撤回している。(pp. 139-140)
\item
  既に海外市場への影響力を強めている中国・インドが海外石炭権益獲得にも
  力を入れてきており、日本としても安定調達に向けて対策が求められている。
  (p. 140左)
\item
  広島県豊田郡大崎上島町で行われている大崎クールジェンプロジェクトは、
  福島県の空気吹きIGCCとは異なる酸素吹きIGCCを採用した実証プラントである。
  酸素吹きIGCCは空気吹きIGCCと比較すると、酸素分離装置に動力を要する
  というデメリットはあるものの、CO2の分離・回収が容易であることと、
  ガス化によって得られた水素を活用して石炭ガス化燃料電池複合発電
  (Integrated coal Gasification Fuel cell Combined Cycle; IGFC,
  送電端効率; 約55\% (HHV))への展開が可能であるというメリットがあり、(p. 140右)
\item
  CCUSに関しては、コストが大きな課題となっている。現在CO2の利用・貯留として
  商用ベースで動いているほとんどのプロジェクトは石油増進回収
  (Enhanced Oil Recovery; EOR)に関するものであり、(p. 140右)
\item
  一般的に石炭と比べて木質バイオマスは粉砕性が悪く、既存発電設備を
  そのまま活用する場合、粉砕性の悪い木質バイオマスが粉砕機に蓄積して
  稼働できなくなるため、混焼率は数\%が限界とされている。(p. 141左)
\end{itemize}


\section{各章を1文で要約する}
\begin{itemize}
\item
  世界における石炭の生産と消費の物量に関する概観をした後、
  世界各国での生産と消費の状況を個別に説明した。
\item
  短期のスポット契約が増加する傾向があり、これにより石炭価格が乱高下する
  状況が発生していることを説明した。
\item
  石炭メジャー4社の一般炭と原料炭それぞれの動向を説明し、
  各社の石炭に対する態度の違いを説明した。
\item
  クリーン・コール・テクノロジーの最近の実証プラントについて紹介した。
\end{itemize}


\section{参考文献をたどって他の文献を読んだ記録}
特になし。


\section{本文PDFファイル名}
137\_134.pdf
% 著者が無料で公開している場合に記載する。
(
\url{https://www.jstage.jst.go.jp/article/journalofmmij/134/10/134_137/_article/-char/ja}
)

% 参考文献を印字する。
\bibliography{myreferences}

\end{document}
